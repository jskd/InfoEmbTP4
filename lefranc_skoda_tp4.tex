\documentclass[12pt]{article}

\usepackage[utf8]{inputenc}
\usepackage[T1]{fontenc}
\usepackage[francais]{babel}
\usepackage[top=2cm, bottom=2cm, left=2cm, right=2cm]{geometry}
\usepackage{url}

\hbadness=99999
\title{TD4 - InfoEmb}
\author{Jérôme Skoda, Joaquim Lefranc}
\date{Novembre 2017}

\begin{document}
\fontfamily{cmr}
\maketitle

\section{Estimations}
\textbf{Initiale :} 3h
\textbf{Réel :}

\section{Etapes pour la création/destruction}
\begin{enumerate}
	\item{\textbf{Quel outil pour quelle mesure?}}
		\newline
		Et bien il faut mesurer le total, donc la ramette, puis le diviser par le nombre de feuilles dans la ramette. C'est le même principe avec le temps dans le tp. On peut mesurer un gros bloc d'opérations puis diviser le total par le nombre d'opérations dans le bloc.
		\newline

	\item{\textbf{(Optionnel) Fonctions de mesure de temps sous linux}}
		\newline

	\item{\textbf{Mesure d'opérations en C : résultat des deux mesures}}
		\newline

		Processus

		Source: tempsExecution/processus.c

		Point initial de mesure du temps: Avant la boucle de fork
		Point final de mesure du temps: Après la boucle de fork

    Les mesures prennent le temps de création d'un fork ainsi que l'incrémentation
	  de la variable n_processus. La mesure du temps d'incrementation parasite
		légèrement la mesure mais il s'agit surement de la solution la _IO_FILE_plus
		à mettre en oeuvre.

    Thread

		Source: tempsExecution/thread.c

    Point initial de mesure du temps: Avant la boucle de pthread_create
		Point final de mesure du temps: Après la boucle de pthread_create

    Comme pour la messure des processus, l'incrémentation de la variable
		n_thread parasite la mesure.


    Résultat obtenu

		Destop          Laptop

processus 53.939709 ms       86.157382 ms
thread    12.604273 ms        24.614902 ms

avec: taskset -c 0

	\item{\textbf{Répétition de la mesure}}
		\newline

	\item{\textbf{Phénomènes et facteurs qui peuvent influencer la mesure}}
		\newline

	\item{\textbf{Répétition de la mesure}}
		\newline
\end{enumerate}

\section{Changement de contexte}

\section{Comparaison avec d'autres résultats de TP}

\section{Outils de « benchmarking »}


\end{document}
