\documentclass[12pt]{article}

\usepackage[utf8]{inputenc}
\usepackage[T1]{fontenc}
\usepackage[francais]{babel}
\usepackage[top=2cm, bottom=2cm, left=2cm, right=2cm]{geometry}
\usepackage{url}

\hbadness=99999
\title{TD4 - InfoEmb}
\author{Jérôme Skoda, Joaquim Lefranc}
\date{Novembre 2017}

\begin{document}
\fontfamily{cmr}
\maketitle

\section{Estimations}
\textbf{Initiale :} 3h
\textbf{Réel :}

\section{Etapes pour la création/destruction}
\begin{enumerate}
	\item{\textbf{Quel outil pour quelle mesure?}}
		\newline
		Et bien il faut mesurer le total, donc la ramette, puis le diviser par le nombre de feuilles dans la ramette. C'est le même principe avec le temps dans le tp. On peut mesurer un gros bloc d'opérations puis diviser le total par le nombre d'opérations dans le bloc.
		\newline

	\item{\textbf{(Optionnel) Fonctions de mesure de temps sous linux}}
		\newline

	\item{\textbf{Mesure d'opérations en C : résultat des deux mesures}}
		\newline

	\item{\textbf{Répétition de la mesure}}
		\newline

	\item{\textbf{Phénomènes et facteurs qui peuvent influencer la mesure}}
		\newline

	\item{\textbf{Répétition de la mesure}}
		\newline
\end{enumerate}

\section{Changement de contexte}

\section{Comparaison avec d'autres résultats de TP}

\section{Outils de « benchmarking »}


\end{document}
